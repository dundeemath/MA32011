% Options for packages loaded elsewhere
% Options for packages loaded elsewhere
\PassOptionsToPackage{unicode}{hyperref}
\PassOptionsToPackage{hyphens}{url}
\PassOptionsToPackage{dvipsnames,svgnames,x11names}{xcolor}
%
\documentclass[
  letterpaper,
  DIV=11,
  numbers=noendperiod]{scrreprt}
\usepackage{xcolor}
\usepackage{amsmath,amssymb}
\setcounter{secnumdepth}{5}
\usepackage{iftex}
\ifPDFTeX
  \usepackage[T1]{fontenc}
  \usepackage[utf8]{inputenc}
  \usepackage{textcomp} % provide euro and other symbols
\else % if luatex or xetex
  \usepackage{unicode-math} % this also loads fontspec
  \defaultfontfeatures{Scale=MatchLowercase}
  \defaultfontfeatures[\rmfamily]{Ligatures=TeX,Scale=1}
\fi
\usepackage{lmodern}
\ifPDFTeX\else
  % xetex/luatex font selection
\fi
% Use upquote if available, for straight quotes in verbatim environments
\IfFileExists{upquote.sty}{\usepackage{upquote}}{}
\IfFileExists{microtype.sty}{% use microtype if available
  \usepackage[]{microtype}
  \UseMicrotypeSet[protrusion]{basicmath} % disable protrusion for tt fonts
}{}
\makeatletter
\@ifundefined{KOMAClassName}{% if non-KOMA class
  \IfFileExists{parskip.sty}{%
    \usepackage{parskip}
  }{% else
    \setlength{\parindent}{0pt}
    \setlength{\parskip}{6pt plus 2pt minus 1pt}}
}{% if KOMA class
  \KOMAoptions{parskip=half}}
\makeatother
% Make \paragraph and \subparagraph free-standing
\makeatletter
\ifx\paragraph\undefined\else
  \let\oldparagraph\paragraph
  \renewcommand{\paragraph}{
    \@ifstar
      \xxxParagraphStar
      \xxxParagraphNoStar
  }
  \newcommand{\xxxParagraphStar}[1]{\oldparagraph*{#1}\mbox{}}
  \newcommand{\xxxParagraphNoStar}[1]{\oldparagraph{#1}\mbox{}}
\fi
\ifx\subparagraph\undefined\else
  \let\oldsubparagraph\subparagraph
  \renewcommand{\subparagraph}{
    \@ifstar
      \xxxSubParagraphStar
      \xxxSubParagraphNoStar
  }
  \newcommand{\xxxSubParagraphStar}[1]{\oldsubparagraph*{#1}\mbox{}}
  \newcommand{\xxxSubParagraphNoStar}[1]{\oldsubparagraph{#1}\mbox{}}
\fi
\makeatother


\usepackage{longtable,booktabs,array}
\usepackage{calc} % for calculating minipage widths
% Correct order of tables after \paragraph or \subparagraph
\usepackage{etoolbox}
\makeatletter
\patchcmd\longtable{\par}{\if@noskipsec\mbox{}\fi\par}{}{}
\makeatother
% Allow footnotes in longtable head/foot
\IfFileExists{footnotehyper.sty}{\usepackage{footnotehyper}}{\usepackage{footnote}}
\makesavenoteenv{longtable}
\usepackage{graphicx}
\makeatletter
\newsavebox\pandoc@box
\newcommand*\pandocbounded[1]{% scales image to fit in text height/width
  \sbox\pandoc@box{#1}%
  \Gscale@div\@tempa{\textheight}{\dimexpr\ht\pandoc@box+\dp\pandoc@box\relax}%
  \Gscale@div\@tempb{\linewidth}{\wd\pandoc@box}%
  \ifdim\@tempb\p@<\@tempa\p@\let\@tempa\@tempb\fi% select the smaller of both
  \ifdim\@tempa\p@<\p@\scalebox{\@tempa}{\usebox\pandoc@box}%
  \else\usebox{\pandoc@box}%
  \fi%
}
% Set default figure placement to htbp
\def\fps@figure{htbp}
\makeatother


% definitions for citeproc citations
\NewDocumentCommand\citeproctext{}{}
\NewDocumentCommand\citeproc{mm}{%
  \begingroup\def\citeproctext{#2}\cite{#1}\endgroup}
\makeatletter
 % allow citations to break across lines
 \let\@cite@ofmt\@firstofone
 % avoid brackets around text for \cite:
 \def\@biblabel#1{}
 \def\@cite#1#2{{#1\if@tempswa , #2\fi}}
\makeatother
\newlength{\cslhangindent}
\setlength{\cslhangindent}{1.5em}
\newlength{\csllabelwidth}
\setlength{\csllabelwidth}{3em}
\newenvironment{CSLReferences}[2] % #1 hanging-indent, #2 entry-spacing
 {\begin{list}{}{%
  \setlength{\itemindent}{0pt}
  \setlength{\leftmargin}{0pt}
  \setlength{\parsep}{0pt}
  % turn on hanging indent if param 1 is 1
  \ifodd #1
   \setlength{\leftmargin}{\cslhangindent}
   \setlength{\itemindent}{-1\cslhangindent}
  \fi
  % set entry spacing
  \setlength{\itemsep}{#2\baselineskip}}}
 {\end{list}}
\usepackage{calc}
\newcommand{\CSLBlock}[1]{\hfill\break\parbox[t]{\linewidth}{\strut\ignorespaces#1\strut}}
\newcommand{\CSLLeftMargin}[1]{\parbox[t]{\csllabelwidth}{\strut#1\strut}}
\newcommand{\CSLRightInline}[1]{\parbox[t]{\linewidth - \csllabelwidth}{\strut#1\strut}}
\newcommand{\CSLIndent}[1]{\hspace{\cslhangindent}#1}



\setlength{\emergencystretch}{3em} % prevent overfull lines

\providecommand{\tightlist}{%
  \setlength{\itemsep}{0pt}\setlength{\parskip}{0pt}}



 


\usepackage{chemarr}
\usepackage{lipsum}
\usepackage{tikz}
\AtBeginDocument{\thispagestyle{empty}
\begin{tikzpicture}[remember picture,overlay]
\node at (current page.center) {\includegraphics[width=\paperwidth,height=\paperheight,keepaspectratio]{MA32011Cover.png}};
\end{tikzpicture}\clearpage}
\KOMAoption{captions}{tableheading}
\makeatletter
\@ifpackageloaded{tcolorbox}{}{\usepackage[skins,breakable]{tcolorbox}}
\@ifpackageloaded{fontawesome5}{}{\usepackage{fontawesome5}}
\definecolor{quarto-callout-color}{HTML}{909090}
\definecolor{quarto-callout-note-color}{HTML}{0758E5}
\definecolor{quarto-callout-important-color}{HTML}{CC1914}
\definecolor{quarto-callout-warning-color}{HTML}{EB9113}
\definecolor{quarto-callout-tip-color}{HTML}{00A047}
\definecolor{quarto-callout-caution-color}{HTML}{FC5300}
\definecolor{quarto-callout-color-frame}{HTML}{acacac}
\definecolor{quarto-callout-note-color-frame}{HTML}{4582ec}
\definecolor{quarto-callout-important-color-frame}{HTML}{d9534f}
\definecolor{quarto-callout-warning-color-frame}{HTML}{f0ad4e}
\definecolor{quarto-callout-tip-color-frame}{HTML}{02b875}
\definecolor{quarto-callout-caution-color-frame}{HTML}{fd7e14}
\makeatother
\makeatletter
\@ifpackageloaded{bookmark}{}{\usepackage{bookmark}}
\makeatother
\makeatletter
\@ifpackageloaded{caption}{}{\usepackage{caption}}
\AtBeginDocument{%
\ifdefined\contentsname
  \renewcommand*\contentsname{Table of contents}
\else
  \newcommand\contentsname{Table of contents}
\fi
\ifdefined\listfigurename
  \renewcommand*\listfigurename{List of Figures}
\else
  \newcommand\listfigurename{List of Figures}
\fi
\ifdefined\listtablename
  \renewcommand*\listtablename{List of Tables}
\else
  \newcommand\listtablename{List of Tables}
\fi
\ifdefined\figurename
  \renewcommand*\figurename{Figure}
\else
  \newcommand\figurename{Figure}
\fi
\ifdefined\tablename
  \renewcommand*\tablename{Table}
\else
  \newcommand\tablename{Table}
\fi
}
\@ifpackageloaded{float}{}{\usepackage{float}}
\floatstyle{ruled}
\@ifundefined{c@chapter}{\newfloat{codelisting}{h}{lop}}{\newfloat{codelisting}{h}{lop}[chapter]}
\floatname{codelisting}{Listing}
\newcommand*\listoflistings{\listof{codelisting}{List of Listings}}
\usepackage{amsthm}
\theoremstyle{definition}
\newtheorem{example}{Example}[chapter]
\theoremstyle{remark}
\AtBeginDocument{\renewcommand*{\proofname}{Proof}}
\newtheorem*{remark}{Remark}
\newtheorem*{solution}{Solution}
\newtheorem{refremark}{Remark}[chapter]
\newtheorem{refsolution}{Solution}[chapter]
\makeatother
\makeatletter
\makeatother
\makeatletter
\@ifpackageloaded{caption}{}{\usepackage{caption}}
\@ifpackageloaded{subcaption}{}{\usepackage{subcaption}}
\makeatother
\makeatletter
\@ifpackageloaded{tcolorbox}{}{\usepackage[many]{tcolorbox}}
\makeatother
%%%% ---foldboxy preamble ----- %%%%%

\definecolor{fbx-default-color1}{HTML}{c7c7d0}
\definecolor{fbx-default-color2}{HTML}{a3a3aa}

\definecolor{fbox-color1}{HTML}{c7c7d0}
\definecolor{fbox-color2}{HTML}{a3a3aa}

% arguments: #1 typelabelnummer: #2 titel: #3
\newenvironment{fbx}[3]{\begin{tcolorbox}[enhanced, breakable,%
attach boxed title to top*={xshift=1.4pt},
boxed title style={boxrule=0.0mm, fuzzy shadow={1pt}{-1pt}{0mm}{0.1mm}{gray}, arc=.3em, rounded corners=east, sharp corners=west}, colframe=#1-color2, colbacktitle=#1-color1, colback = white, coltitle=black,  titlerule=0mm, toprule=0pt, bottomrule=.7pt, leftrule=.3em, rightrule=0pt, outer arc=.3em,  arc=0pt,	 sharp corners = east, left=.5em, bottomtitle=1mm, toptitle=1mm,title=\textbf{#2}\hspace{0.5em}{#3}]}
{\end{tcolorbox}}

% boxed environment with right border
\newenvironment{fbxSimple}[3]{\begin{tcolorbox}[enhanced, breakable,%
attach boxed title to top*={xshift=1.4pt},
boxed title style={boxrule=0.0mm, fuzzy shadow={1pt}{-1pt}{0mm}{0.1mm}{gray}, arc=.3em, rounded corners=east, sharp corners=west}, colframe=#1-color2, colbacktitle=#1-color1, colback = white, coltitle=black,  titlerule=0mm, toprule=0pt, bottomrule=.7pt, leftrule=.3em, rightrule=.7pt, outer arc=.3em,  	left=.5em, right=.5em, bottomtitle=1mm, toptitle=1mm,title=\textbf{#2}\hspace{0.5em}{#3}]}
{\end{tcolorbox}}

%%%% --- end foldboxy preamble ----- %%%%%
%%==== colors from yaml ===%
\definecolor{TODO-color1}{HTML}{e7b1b4}
\definecolor{TODO-color2}{HTML}{8c3236}
\definecolor{DONE-color1}{HTML}{cce7b1}
\definecolor{DONE-color2}{HTML}{86b754}
\definecolor{Proof-color1}{HTML}{c0c0c0}
\definecolor{Proof-color2}{HTML}{808080}
\definecolor{Conjecture-color1}{HTML}{948bde}
\definecolor{Conjecture-color2}{HTML}{584eab}
\definecolor{Solution-color1}{HTML}{c0c0c0}
\definecolor{Solution-color2}{HTML}{808080}
\definecolor{Corollary-color1}{HTML}{948bde}
\definecolor{Corollary-color2}{HTML}{584eab}
\definecolor{Lemma-color1}{HTML}{948bde}
\definecolor{Lemma-color2}{HTML}{584eab}
\definecolor{Definition-color1}{HTML}{d999d3}
\definecolor{Definition-color2}{HTML}{a01793}
\definecolor{Theorem-color1}{HTML}{948bde}
\definecolor{Theorem-color2}{HTML}{584eab}
\definecolor{Feature-color1}{HTML}{c0c0c0}
\definecolor{Feature-color2}{HTML}{808080}
%=============%
\usepackage{bookmark}
\IfFileExists{xurl.sty}{\usepackage{xurl}}{} % add URL line breaks if available
\urlstyle{same}
\hypersetup{
  pdftitle={MA32011},
  pdfauthor={Philip Murray},
  colorlinks=true,
  linkcolor={blue},
  filecolor={Maroon},
  citecolor={Blue},
  urlcolor={Blue},
  pdfcreator={LaTeX via pandoc}}


\title{MA32011}
\author{Philip Murray}
\date{2026-01-19}
\begin{document}
\maketitle

\renewcommand*\contentsname{Table of contents}
{
\hypersetup{linkcolor=}
\setcounter{tocdepth}{2}
\tableofcontents
}

\bookmarksetup{startatroot}

\chapter*{Preface}\label{preface}
\addcontentsline{toc}{chapter}{Preface}

\markboth{Preface}{Preface}

Welcome to the module MA32011 Dynamical systems.

My name is Philip Murray and I am the module lead.

\section*{How to contact me?}\label{how-to-contact-me}
\addcontentsline{toc}{section}{How to contact me?}

\markright{How to contact me?}

\begin{itemize}
\tightlist
\item
  email: pmurray@dundee.ac.uk
\item
  office: G11, Fulton Building
\item
  Teams: PM me
\end{itemize}

\section*{Lecture notes}\label{lecture-notes}
\addcontentsline{toc}{section}{Lecture notes}

\markright{Lecture notes}

You can find lecture notes for the module on this page. If you would
like a pdf this can be easily generated by clicking on the pdf link of
the webpage. I will occasionally edit/update the notes as we proceed
through lectures. If you spot any errors, typos or omissions please let
me know.

\section*{Reading}\label{reading}
\addcontentsline{toc}{section}{Reading}

\markright{Reading}

Nonlinear dynamics and chaos, Steven Strogatz Strogatz (2001)
Mathematical Biology I, Murray (2002)

\section*{Python codes}\label{python-codes}
\addcontentsline{toc}{section}{Python codes}

\markright{Python codes}

I have provided Python codes for most of the figures in the notes (you
can unfold code section by clicking `Code'). Note that the Python code
does not appear in the pdf.

Many of you have taken the Introduction to Programming module at Level 2
and have therefore some experience using Python. I strongly encourage
you to use the provided codes as a tool to play around with numerical
solutions of the various models that we will be working on. The codes
should run as standalone Python codes.

\begin{tcolorbox}[enhanced jigsaw, arc=.35mm, titlerule=0mm, colbacktitle=quarto-callout-note-color!10!white, leftrule=.75mm, rightrule=.15mm, coltitle=black, breakable, toprule=.15mm, opacityback=0, colframe=quarto-callout-note-color-frame, toptitle=1mm, bottomrule=.15mm, bottomtitle=1mm, title=\textcolor{quarto-callout-note-color}{\faInfo}\hspace{0.5em}{Note}, left=2mm, opacitybacktitle=0.6, colback=white]

To access Python on Uni machines:

\begin{enumerate}
\def\labelenumi{\arabic{enumi}.}
\tightlist
\item
  Launch Anaconda from AppsAnywhere
\item
  When a folder opens, double click on \emph{Spyder}.
\item
  Paste a code from lecture notes into the editor on the left-hand side.
\item
  Click on the green arrow to run the code.
\item
  The plots should appear in the plots tab on the right-hand side.
\item
  Experiment with the code. When you change a model parameter, does the
  solution change in an expected way?
\end{enumerate}

\end{tcolorbox}

I have also provided some examples of how to use Python as a symbolic
calculator. This uses a Python library called \emph{sympy} and is quite
similar to Maple.

\section*{Assessment}\label{assessment}
\addcontentsline{toc}{section}{Assessment}

\markright{Assessment}

\begin{itemize}
\item
  Final exam (80 \%)
\item
  2 class tests (8 \% each), Week 7 and 11
\item
  4 quizes (1 \% each), Week 2,4,6 and 9
\end{itemize}

\section*{Plan}\label{plan}
\addcontentsline{toc}{section}{Plan}

\markright{Plan}

\begin{longtable}[]{@{}llll@{}}
\caption{Projected delivery}\tabularnewline
\toprule\noalign{}
Week & Up to Section & Tutorial sheet & Assessment \\
\midrule\noalign{}
\endfirsthead
\toprule\noalign{}
Week & Up to Section & Tutorial sheet & Assessment \\
\midrule\noalign{}
\endhead
\bottomrule\noalign{}
\endlastfoot
1 & & 1 & \\
2 & & 1 & Quiz 1 \\
3 & & 2 & \\
4 & & 2 & Quiz 2 \\
5 & & 3 & \\
6 & & 3 & Quiz 3 \\
7 & & 4 & Test 1 \\
8 & & 4 & \\
9 & & 5 & Quiz 4 4 \\
10 & & 5 & \\
11 & & Test 2 & \\
\end{longtable}

\section*{References}\label{references}
\addcontentsline{toc}{section}{References}

\markright{References}

\bookmarksetup{startatroot}

\chapter{Introduction}\label{introduction}

The goal of this module is to provide an introduction to dynamical
systems. We will introduce key mathematical concepts and explore
examples from physics and biology.

To begin with we introduce some key terminology.

\section{Discrete v continuous time}\label{discrete-v-continuous-time}

\subsection{Differential equations}\label{differential-equations}

Let \(t\) be a continuous variable and \(x=x(t)\). Consider the ordinary
differential equation (ODE)
\begin{equation}\phantomsection\label{eq-logdemo}{
\dot{x}=rx(1-x).
}\end{equation}

\(\dot{x}\) is used to denote the time derivative, i.e. \[
\dot{x}:=\frac{dx}{dt}.
\]

Similarly, the second order derivative is represented by \[
\ddot{x}:=\frac{d^2x}{dt^2}.
\]

Many of the second order ODE problems that we will examine originate
from Newton's Second Law, i.e.

\[
m\ddot{x}=F(x)
\] where \(x(t)\) represents the position of a particle at time, \(t\),
\(m\) represents a constant particle mass and \(F\) a resultant force.

Consider the case in which \(F\) represents a linear restoring force,
i.e. \[
F(x)=-kx.
\] The equation of motion can be written as
\begin{equation}\phantomsection\label{eq-secondorderODE}{
\ddot{x}=-\mu x,
}\end{equation} where \(\mu=-k/m\).

\begin{example}[]\protect\hypertarget{exm-}{}\label{exm-}

The app in Figure~\ref{fig-popmodel} encodes a numerical solution of the
second order ODE

\[
a\ddot{x}+b\dot{x}+cx=0.
\]

Choose an appropriate value of the parameters \(a\), \(b\) and \(c\) so
that the solution captures the case of a particle of mass (\(m\)) equal
to 3 subjected to a linear restoring force with spring constant (\(k\))
equal to 5.

Is the behaviour of the numerical solution consistent with
Equation~\ref{eq-secondorderODE}.

\end{example}

\subsection{Difference equations}\label{difference-equations}

Suppose that \(n\) is a discrete variable. Let \(y_n\) represent a
dependent variable at iteration \(n\). Consider the difference equation
\[
y_{n+1}=ry_n(1-y_n),
\] where \(r\in \Re\).

See this
\href{https://dundeemath.github.io/Admissions/posts/RecurrenceRelations.html}{link}
for exploration of model solutions.

\subsection{Key questions to ask of a dynamical
system}\label{key-questions-to-ask-of-a-dynamical-system}

\begin{itemize}
\tightlist
\item
  do solutions exist? If so are they unique?
\item
  Is there an explicit solution?
\item
  Can we qualitatively describe solution behaviour?
\item
  How do the solutions depend on the model parameters?
\item
  Are their critical values of parameters where solution behaviour
  changes?
\end{itemize}

\section{Autonomous v nonautonomous
ODEs}\label{autonomous-v-nonautonomous-odes}

For an autonomous system, the update does not explicitly depend on the
independent variable. Equation~\ref{eq-logdemo} is autonomous. But \[
\dot{x}=rx(1-x+t)
\] is nonautonomous (because of the explicit time dependence on the
right-hand side).

\section{Linear v Nonlinear}\label{linear-v-nonlinear}

Linear systems satisfy a linear supposition principle: a sum of
solutions is itself a solution. In general, this property does not hold
for nonlinear systems.

In linear dynamical systems, the dynamics are a function of linear sums
of the dependent variables. Hence

\[
\dot{x}=-x
\] is a linear ordinary differential equation (ODE). But \[
\dot{x}=-x^2
\] is nonlinear.

\section{Quantitative v qualitative
solutions}\label{quantitative-v-qualitative-solutions}

You are likely used to solving problems in which an explicit solution
can be found. For example, consider the ODE \[
\dot{x}=-kx, \quad  x(0)=x_0
\] where \(k,x_0 \in \Re^+\) .

We can integrate and express the solution as

\[
x(t)=x_0e^{-kt}
\]

Using the explicit solution we can then answer questions about its
behaviour. For example, let's say we want to find the time, \(t^*\), at
which the solution is half it's maximum. Hence \[
x(t*)=x_0/2 \implies t^* = \frac{\ln 2}{k}.
\]

However, in the study of nonlinear systems, most problems will not have
an explicit solution. For example, consider the nonlinear ODE

\[
\dot{x}=-\frac{k\sin(x)+\sqrt{x}}{1+x}, \quad  x(0)=x_0
\] where \(0<x_0<\pi\).

I cannot integrate this equation in order to find solutions in terms of
standard functions. Hence I cannot \emph{quantitatively} describe the
solution. However, I can identify that

\[
\dot{x}<0, \  \forall \  0<x<\pi.
\]

Hence the solution will decrease in value from the given initial
condition and tend to zero as \(t\rightarrow \infty\). This is an
example of a \emph{qualitative} analysis.

\section{Representing solutions}\label{representing-solutions}

It is useful to introduce some nomenclature to describe the solutions of
a dynamical system. Consider an ODE \[
\dot{x}=f(x), \quad x(0)=x_0,
\] where \(f\) is a prescribed function and \(x_0\) is an initial
condition.

\begin{itemize}
\tightlist
\item
  phase space - put dependent variable on Cartesian axes
\item
  \emph{phase point} - value of the solution at given time point
\item
  \emph{vector field} - the derivative of the solution, i.e.~\(f\).
\item
  \emph{trajectory} - a line in phase space that traces out a solution
  as time evolves
\item
  \emph{phase portrait} - collection of trajectories (i.e.~solutions
  with different initial conditions)
\end{itemize}

\section{Fixed points and their
stability}\label{fixed-points-and-their-stability}

Many of the dynamical systems that we will study will be nonlinear.
Hence it will not be possible to compute exact solutions.

The behaviour of dynamical systems can often be understood by
considering the fixed points, i.e.~values of the dependent variables at
which the dynamics are at steady state.

Stability analyses are used to investigate the dynamics of perturbations
about the steady state.

\section{Uniqueness and existence}\label{uniqueness-and-existence}

We will restrict ourselves to problems in which the vector fields are
sufficiently well behaved such that unique solutions exist. However,
problems can be identified where solutions do not exists or where
multiple solutions exist.

\begin{example}[]\protect\hypertarget{exm-}{}\label{exm-}

Show that the solution to the ODE \[
\dot{x}=x^2 \quad x(0)=2
\] is not defined for all \(t>0\).

\end{example}

\begin{example}[]\protect\hypertarget{exm-}{}\label{exm-}

Show that the solution to the ODE \[
\dot{x}=\sqrt{x} \quad x(0)=0
\] is given by \[
x=\begin{cases}
&0 \ \  &t\leq\delta \\
&\frac{(t-\delta)^2}{4}>\delta \ &t> \delta
\end{cases}
\]

\end{example}

\section{Nondimensionalisation}\label{nondimensionalisation}

In real world problems, variables and parameters typically have units
(e.g.~time - seconds, Force - Newtons etc.). We can nondimensionalise
problems by defining rescaled variables. This process can be used to
justify simplifications to models and to reduce the number of
parameters.

\section{Numerical solutions}\label{numerical-solutions}

Numerical solutions are used to numerically compute approximate
solutions to problems. The simplest example of a numerical method in a
dynamical system is the forward Euler method. Suppose we want to study
the ODE

\[
\dot{x}=f(x), \quad x(0)=x_0, \quad 0<t<T.
\]

Discretise the independent variable \(t\) by defining
\(t=0\),\(\Delta t\),\(2\Delta t\),\ldots,\(T=N\Delta T\).

Approximate the time derivative

\[
\dot{x}=\frac{dx}{dt}\sim \frac{x(t+\Delta t)-x(t)}{\Delta t}
\]

Hence the solution at time \(t+\Delta t\) can be approximated by \[
x(t+\Delta t)=x(t)+\Delta t f(x(t)).
\]

Given an initial condition \(x(0)=a\) we can compute the approximate
solution at time \(\Delta t\). Further iteration then allows an
approximate solution to be calculated.

Numerical solutions provide a a very useful way to explore solution
behaviour. However, they describe the quantitative behaviour of a
solution for a particular initial condition and set of parameter values.

\part{1D flows}

\chapter{Flows on the line}\label{flows-on-the-line}

First order in time.

Let \(x=x(t)\). \begin{equation}\phantomsection\label{eq-fp1d}{
\dot{x}=f(x).
}\end{equation}

\(f\) is smooth and real valued. autonomous. Nonlinear.

\section{Geometric}\label{geometric}

\begin{example}[]\protect\hypertarget{exm-}{}\label{exm-}

Let \(x(t)\). Consider the ODE \[
\dot{x}=\sin x.
\]

For the initial condition \(x(0)=\pi/4\), describe solution behaviour as
\(t\rightarrow \infty\).

\end{example}

\phantomsection\label{Solutionux2a-1}
\begin{fbx}{Solution}{Solution}{}
\phantomsection\label{Solution*-1}

An implicit solution is

\[
t=-\ln |\csc x + \cot x| + C,
\] where \(C\) is an integration constant.

\pandocbounded{\includegraphics[keepaspectratio]{1DFlows_files/figure-pdf/cell-2-output-1.pdf}}

\end{fbx}

\section{Fixed points and stability}\label{fixed-points-and-stability}

\subsection{Fixed points}\label{fixed-points}

Let \(x=x^*\) be a fixed point of Equation~\ref{eq-fp1d}. At \(x=x^*\)
\[
\dot{x}=0 \implies f(x^*)=0.
\]

There are a number of interpretations of \(x^*\):

\begin{itemize}
\tightlist
\item
  roots of \(f\) (algebraic)
\item
  stagnation points of the flow (topological)
\end{itemize}

\phantomsection\label{Corollary-1}
\begin{fbxSimple}{Corollary}{Corollary 1}{}
\phantomsection\label{Corollary-1}
Any trajectory initialised at a fixed point remains there for all \(t\).

\end{fbxSimple}

\begin{example}[]\protect\hypertarget{exm-}{}\label{exm-}

Find all the fixed points of \[
\dot{x}=x^2-1,
\]

\end{example}

\phantomsection\label{Solutionux2a-2}
\begin{fbx}{Solution}{Solution}{}
\phantomsection\label{Solution*-2}

The fixed points are \(x^*=\pm 1\).

\pandocbounded{\includegraphics[keepaspectratio]{1DFlows_files/figure-pdf/cell-3-output-1.pdf}}

\end{fbx}

\subsection{Linear stability analysis}\label{linear-stability-analysis}

\subsubsection{A change of dependent
variable}\label{a-change-of-dependent-variable}

To perform a linear stability analysis we make the change of variables
\[
x(t)=x^*+\hat{x}(t)
\] where the new dependent variable, \(\hat{x}(t)\), is a perturbation
about the steady state.

The time derivative on the left-hand side of
\textbf{?@eq-gensinglespecodeagain} transforms to \[
\dot{x}= \frac{d }{dt} (x^*) + \frac{d }{dt}(\hat{x}(t))=\dot{\hat{x}}.
\] Hence \textbf{?@eq-gensinglespecodeagain} transforms to \[
\dot{\hat{x}} = f(x^*+\hat{x}(t)).
\]

\subsubsection{Taylor expansion and a linear
system}\label{taylor-expansion-and-a-linear-system}

Employing the Taylor expansion on the right-hand side of
\textbf{?@eq-gensinglespecodeagain} and making the assumption that
perturbations are small \[
\dot{\hat{x}} = f(x^*)+  f'(x^*)\hat{x}(t) + f''(x^*)\hat{x}^2(t) + h.o.t.
\] Noting that\\
\[
f(x^*)=0
\] and retaining linear terms yields \[
\dot{\hat{x}} =  f'(x^*)\hat{x}(t)
\] with solution \[
\hat{x}(t)=  \eta e^{f'(x^*) t}
\] where \(\eta\) is some initial perturbation about the steady-state.

\subsubsection{A condition for linear
stability}\label{a-condition-for-linear-stability}

When \(f'(x^*)>0\) the perturbation grows exponentially fast and the
steady-state is unstable. When \(f'(x^*)<0\) the perturbation decays
exponentially fast and the steady-state is stable.

\begin{example}[]\protect\hypertarget{exm-}{}\label{exm-}

Determine the linear stability of the fixed points of \[
\dot{x}=x^2-1.
\]

\end{example}

\begin{example}[]\protect\hypertarget{exm-}{}\label{exm-}

What can be said about the stability of the fixed points of the
following ODEs: 1. \[
\dot{x}=-x^3.
\] 2. \[
\dot{x}=x^3.
\]

\end{example}

\section{Validity of linear
classification}\label{validity-of-linear-classification}

Hyperbolic FP - eigenvalues nonzero Hartman Grobman - if system has
hyperbolic FP, classification of nonlinear system determined by linear
classification non-hypberbolic FP - need to consider higher order terms.

\section{Case study: population
dynamics}\label{case-study-population-dynamics}

Let \(N=N(t)\). The logistic model of population growth, due to
Verhulst, takes the form
\begin{equation}\phantomsection\label{eq-logisticode}{
\dot{N}=rN(t)\left (1-\frac{N(t)}{K}\right),
}\end{equation}

where \(r\) is the linear growth rate and \(K\) is carrying capacity. We
consider both \(r,K\in \Re^+\).

Questions to ask of such a model are: what type of biologically
realistic solutions does it possess? Are there steady-states? If so, are
they stable or unstable? Are there bifurcations in solutions?

\subsubsection{Numerical solutions}\label{numerical-solutions-1}

In Figure~\ref{fig-logisticgrowthmodel} we present numerical solutions
of equation using different initial conditions. Note the limiting
behaviour of solutions as \(t\rightarrow \infty\). In
Figure~\ref{fig-logisticgrowthmodel} it is clear that even though some
solutions are initialised at \(N_0=0.1\), much closer to \(N^*=0\) than
\(N^*=K\), they tend to the limit \(N=K\). Why do solutions not tend to
\(N^*=0\)?

\begin{figure}

\centering{

\pandocbounded{\includegraphics[keepaspectratio]{1DFlows_files/figure-pdf/fig-logisticgrowthmodel-output-1.pdf}}

}

\caption{\label{fig-logisticgrowthmodel}Numerical solution of the
logistic growth model}

\end{figure}%

\subsubsection{Dimensional analysis and
nondimensionalisation}\label{dimensional-analysis-and-nondimensionalisation}

\(N\) represents the population density and has units of one over area
(say \(1/m^2\)) and \(t\) has units of time (say, seconds, \(s\)). Hence
the left-hand side of Equation~\ref{eq-logisticode} has units of
\(1/(m^2 s)\). The first term on the right-hand side of
Equation~\ref{eq-logisticode} is \(rN\). \(N\) has units \(1/m^2\) hence
the parameter \(r\) must have units of \(1/s\) for dimensional
consistency. This is consistent as \(r\) represents the linear growth
rate.

The second term has the form \(rN^2/K\). Given the chosen units for
\(r\) and \(N\), the parameter \(K\) must have dimensions \(1/m^2\).
Again, this is consistent as \(K\) is a carrying capacity (i.e.~it has
units of population density).

We define the nondimensionalised variables \[
n=\frac{N}{\tilde{N}} \ \ \ \ \ \ \tau=\frac{t}{\tilde{T}}
\] where \(\tilde{N}\) and \(\tilde{T}\) are constants that have units
of population density and time, respectively. Hence
Equation~\ref{eq-logisticode} transforms, upon change of variables, to
\[
\begin{aligned}
\frac{\tilde{N}}{\tilde{T} }\frac{dn}{d\tau}=r\tilde{N}n(1-\frac{n\tilde{N}}{K}).
\end{aligned}
\]

In the case of the logistic equation there is only one time scale and
density scale in the problem, hence we choose \[
\tilde{T}=\frac{1}{r}  \ \ \ and \ \ \ \tilde{N}=K
\] and the dimensionless model is
\begin{equation}\phantomsection\label{eq-logisticnondim}{
\begin{aligned}
\frac{dn}{d\tau}= n(1-n)
\end{aligned}
}\end{equation} Note that we can retrieve the original equation by
rescaling and calculating \(N=\tilde{N}n\) and \(t=\tilde{T}\tau\).

\subsection{Steady states and linear
stability}\label{steady-states-and-linear-stability}

Steady states satisfy \[
n^*(1-n^*)=0.
\] Hence \[
n^*=0, \ \ \ \ n^*=1.
\]

To determine linear stability we compute \[
H'(n)= (1-2n).
\] When \(n=n^*=0\) we obtain \[
H'(n)= 1.
\] Hence the origin is an unstable steady-state.

At the steady-state \(n^*=1\) \[
H'(n^*)= -1
\] hence \(n^*=1\) is linearly stable.

Note that the linear stability analysis can explain the observations
regarding the numeric solutions presented in
Figure~\ref{fig-logisticgrowthmodel}.

\subsection{Graphical analysis}\label{graphical-analysis}

In Figure~\ref{fig-dlogisticrhs} we plot the right-hand side of
Equation~\ref{eq-logisticnondim}. We can qualitatively describe model
solutions by considering the arrow along the x axis. Suppose we consider
an initial condition with \(0<n_0<1\). Using the graph of \(H(n)\),
\(dn/d\tau\) is positive, hence \(n\) increases as a function of time
until \(n(\tau)\rightarrow 1\).

\begin{figure}

\centering{

\pandocbounded{\includegraphics[keepaspectratio]{1DFlows_files/figure-pdf/fig-dlogisticrhs-output-1.pdf}}

}

\caption{\label{fig-dlogisticrhs}Right-hand side of the logistic ODE}

\end{figure}%

\subsection{An exact solution}\label{an-exact-solution}

Separation of variables yields \[
\int\frac{ dN}{N(1-\frac{N}{K})}=r\int dt.
\] Using partial fractions \[
\int\frac{ dN}{N} + \frac{1}{K}\int\frac{ dN}{1-\frac{N}{K}}=r\int dt.
\] Integration yields \[
\ln N - \ln\left(1-\frac{N}{K}\right)= \ln \frac{N }{1-\frac{N}{K}} =  rt+C.
\] Hence \[
N=\frac{De^{rt}}{1+\frac{D}{K}e^{rt}}
\] Given an initial condition \(N(0)=N_0\), we obtain \[
N(t)=\frac{N_0K e^{rt}}{K+N_0(e^{rt}-1)}
\]

\subsubsection{Qualitative analysis of the exact
solution}\label{qualitative-analysis-of-the-exact-solution}

As \(t\rightarrow \infty\), \(N\rightarrow K\). At \(t=0\), \(N=N_0\)
and that for small \(N_0\ll K\) the initial growth phase is exponential,
i.e.~ \[
N(t)\sim N_0 e^{rt} \\ \ \ \ \ N_0\ll K, t\ll \frac{1}{r}.
\]

Note that in almost all the models that we will consider the above
method is not an usually an option as the ODE is not explicitly
integrable.

\section{Existence and uniqueness}\label{existence-and-uniqueness}

Consider the ODE \[
\dot{x}=f(x), \quad x(a)=b.
\] There exists a unique solution on a local interval \(x\in[c,d]\) with
\(c<x_0<d\) given that \(f\) is bounded and Lipschitz continuous on
\([c,d]\).

\begin{example}[]\protect\hypertarget{exm-}{}\label{exm-}

Show that

\[
\dot{x}=x^{\frac{1}{2}}, \ \quad x(0)=0
\] has an infinite family of solutions.

\end{example}

\section{Impossibility of
oscillations}\label{impossibility-of-oscillations}

\section{Potential flows}\label{potential-flows}

Consider the ODE \[
\dot{x}=f(x).
\]

Suppose that \[
f(x)=-\frac{d V(x)}{dx}.
\]

Now consider \[
\dot{V}.
\] Applying the chain rule \[
\dot{V}=\frac{dV}{dx}\dot{x}=-(\frac{dV}{dx})^2\leq 0.
\]

Hence for a potential flow \(V\) is never increasing.

\section{Numerical solutions}\label{numerical-solutions-2}

\section{References}\label{references-1}

\part{Appendices}

\chapter{Python}\label{python}

\section{Symbolic calculations}\label{symbolic-calculations}

Symbolic calculations ahve been performed using the Python library
\href{https://docs.sympy.org/latest/index.html}{Sympy}.

This library comes with
\href{https://docs.sympy.org/latest/tutorials/index.html\#tutorials}{tutorials}.

You are encouraged to familiarise yourself with the syntax by working
through some of the tutorial examples provided at the links above.

Many of the calculations that we do throughout the course involve
solving systems of algebraic equations

\section{Numerical solution of difference
equations}\label{numerical-solution-of-difference-equations}

Difference equations have been solved using a for loop. Routinse have
been written to solve either single or coupled system of difference
equaitons.

\section{Numerical integration of
ODEs}\label{numerical-integration-of-odes}

Throughout the notes systems of ODEs have been integrated using the
\href{https://scipython.com}{Scipy} function
\href{https://docs.scipy.org/doc/scipy/reference/generated/scipy.integrate.odeint.html}{odeint}.

\section{Plotting}\label{plotting}

Line graphs are plotted using the Python library
\href{https://matplotlib.org}{Matplotlib}.

\phantomsection\label{refs}
\begin{CSLReferences}{1}{0}
\bibitem[\citeproctext]{ref-murray2002mathematical}
Murray, J. D. 2002. \emph{Mathematical Biology i: An Introduction}.
Springer.

\bibitem[\citeproctext]{ref-strogatz2001nonlinear}
Strogatz, Steven H. 2001. \emph{Nonlinear Dynamics and Chaos: With
Applications to Physics, Biology, Chemistry, and Engineering (Studies in
Nonlinearity)}. Vol. 1. Westview press.

\end{CSLReferences}




\end{document}
